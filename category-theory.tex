\documentclass[]{article}
\usepackage{amsmath,amscd}
\usepackage{amssymb}
\usepackage{amsthm}
\usepackage{todonotes}
\usepackage{tikz-cd}
\usepackage{calligra}
\DeclareMathAlphabet{\catty}{T1}{calligra}{m}{n}

\usepackage [english]{babel}
\usepackage [autostyle, english = american]{csquotes}
\usepackage{fancyvrb}
\DefineVerbatimEnvironment{code}{Verbatim}{fontsize=\small}
\MakeOuterQuote{"}

\newtheorem{theorem}{Theorem}
\newtheorem{definition}{Definition}
\newtheorem{exercise}{Exercise}

\newcommand{\what}{}
% \newcommand{\cat}[1]{\mathbf{#1}}
\newcommand{\defn}[2]{
\renewcommand{\what}{\textit{#1} }
\textbf{Definition:} #2\\
}
\newcommand{\diag}[1]{$$\begin{CD}#1\end{CD}$$}
\newcommand{\cdr}[1]{\arrow[swap]{r}{#1}}
\newcommand{\cdrn}[1]{\arrow[swap,dotted]{r}{#1}}
\newcommand{\cdrr}[1]{\arrow[swap]{rr}{#1}}
\newcommand{\cddl}[1]{\arrow{dl}{#1}}
\newcommand{\cddr}[1]{\arrow[swap]{dr}{#1}}
\newcommand{\cdl}[1]{\arrow{l}{#1}}
\newcommand{\cdd}[1]{\arrow{d}{#1}}
\newcommand{\cdu}[1]{\arrow{u}{#1}}
\newcommand{\mono}{\rightarrowtail}
\newcommand{\epi}{\twoheadrightarrow}

\newcommand{\abs}[1]{\left|#1\right|}
\newcommand{\setn}[1]{\left\{#1\right\}}
\renewcommand{\hom}[2]{\ensuremath{\text{Hom}(#1,\;#2)}}

\newcommand{\tfarr}[4][\to]{\ensuremath{#2 : #3 #1 #4}}
\newcommand{\functor}[3]{\ensuremath{#1 : \cat{#2} \to \cat{#3}}}
\newcommand{\cat}[1]{\ensuremath{\!\! \catty{#1} \,\,}}

%opening
\title{Category Theory}
\author{Sandy Maguire}

\begin{document}

\maketitle

\begin{abstract}
My notes summarizing Awodey for the purposes of learning Category Theory. It's
going to be a great project.
\end{abstract}

\newpage

\tableofcontents

\newpage

\section{Foundations}

\subsection{Definition of a Category}

A category \cat{C} consists of \textbf{objects} and \textbf{arrows} between
them. To be precise, every arrow has a domain and a codomain, both of which are
objects in the category. In addition, every object $X \in \cat{C}$ has an
\textbf{identity arrow} \tfarr{1_X}{X}{X}.

If \tfarr{f}{A}{B}, we say $dom(f) = A$ and $cod(f) = B$.

In addition, to be a category, \cat{C} must respect the following laws:

\begin{enumerate}
  \item{Composition: If \tfarr{f}{A}{B} and \tfarr{g}{B}{C}, there exists an
    arrow \tfarr{g \circ f}{A}{C}.}
  \item{Associative: $f \circ (g \circ h) = (f \circ g) \circ h$}.
  \item{Identity: $1_B \circ f = f = f \circ 1_A$, given \tfarr{f}{A}{B}}
\end{enumerate}

Anything that satisfies these laws is a category. It need not correspond to our
intuitions that "arrows are functions" or any such silliness.

\subsection{Definition of a Functor}

A functor \functor{F}{C}{D} is a mapping of objects in \cat{C} to objects in
\cat{D}, and likewise for arrows. A functor is a homomorphism across domains,
codomains and compositions.

That is to say:

\begin{enumerate}
  \item{$F(\tfarr{f}{A}{B}) = \tfarr{F(f)}{F(A)}{F(B)}$}
  \item{$F(g\circ f) = F(g) \circ F(f)$}
  \item{$F(1_A) = 1_{F(A)}$}
\end{enumerate}

There is an identity functor \functor{1_{\cat{C}}}{C}{C}, and because functors
compose, we have a category of categories: \cat{Cat}.

\subsection{Definition of an Isomorphism}

An arrow \tfarr{f}{A}{B} is called an isomorphism if there exists an arrow
\tfarr{g}{B}{A} such that $g \circ f = 1_A$ and $f \circ g = 1_B$.

\begin{theorem}
Isomorphisms are unique.
\begin{proof}
  For \tfarr{f}{A}{B} to be an isomorphism, we must have \tfarr{g}{B}{A}. Assume
  that the isomorphism is not unique, and thus that we also have
  \tfarr{g'}{B}{A}.

  By definition, we have $g \circ f = 1_A$. We can compose on both sides to get
  $g \circ f \circ g' = 1_A \circ g'$, but recall that $g'$ is an isomorphism,
  therefore $g \circ 1_B = 1_A \circ g'$. We can emit the identities, and thus
  $g = g'$.
\end{proof}
\end{theorem}

\begin{definition}
  A \textbf{group} is a single object category where every arrow is an
  isomorphism.
\end{definition}

\begin{theorem}
Every group $G$ is isomorphic to a group of permutations.
\begin{proof}
  Define $f_g(x) = g \times x$ for $g \in G$. Since $G$ is a group, we also have
  $f_{g^{-1}}(x) = g^{-1} \times x = f_g^{-1}(x)$ which means $(f_g \circ
  f_g^{-1})(x) = (f_g^{-1} \circ f_g)(x) = x$. Therefore $f_g$ forms a group.

  Consider now a function $\tfarr{T}{G}{\bar{G}}$ where $T(g) = f_g$.  $T$ is a
  group homomorphism because $(f_g \circ f_h)(x) = f_g(f_h(x)) = f_g(h * x) = g
  * (h * x) = f_{g*h}(x)$.

  This is true for any $x$, therefore $T(g) \circ T(h) = f_g \circ f_h = f_{g*h}
  = T(g * h)$
\end{proof}
\end{theorem}

\section{Constructions on Categories}

\subsection{Product Category}

The product of categories \cat{C} and \cat{D} is $\cat{C}\times\cat{D}$. It's
objects are the cartesian product of objects in \cat{C} and \cat{D}. Arrows are
likewise defined in this matter, with composition and units being defined
component-wise:

\begin{align*}
  1_{(C, D)} &= (1_C, \; 1_D) \\
  (f', g') \circ (f, g) &= (f' \circ f, \; g' \circ g)
\end{align*}

There are also projection functors
$\functor{\pi_1}{\cat{C}\times\cat{D}}{\cat{C}}$ and
$\functor{\pi_2}{\cat{C}\times\cat{D}}{\cat{C}}$ defined in the obvious way.

\subsection{Arrow Category}

The arrow category $\cat{C}^\rightarrow$ has objects which are arrows in \cat{C}
and its arrows are commutative squares in \cat{C}.

For example, given $f,\;f',\;g_1,\;g_2 \in \cat{C}$, there is an arrow $\tfarr{g
}{f}{f'} \in \cat{C}^\rightarrow$ such that $g = (g_1,\;g_2)$ if the follow
diagram exists in \cat{C}:

$$\begin{tikzcd}
  A \cdr{g_1} \cdd{f} & A' \cdd{f'} \\
  B \cdr{g_2} & B'
\end{tikzcd}$$

Composition acts as you'd expect. Given $h \circ g$, we get the diagram:

$$\begin{tikzcd}
  A \cdr{g_1} \cdd{f} & A' \cdr{h_1} \cdd{f'} & A'' \cdd{f''}  \\
  B \cdr{g_2} & B' \cdr{h_2} & B''
\end{tikzcd}$$

In other words, there is an arrow $f\to f'$ iff $g_2\circ f = f' \circ g_1$. But
what does this mean?

In a monoid, composition means concatenation, and thus over a free monoid of the
alphabet, $f = \text{art}$, $f' = \text{far}$, $g = (\text{t}, \text{f})$
because $\text{f}\circ\text{art} = \text{far}\circ\text{t}$. With the monoid of
addition over the natural numbers, the arrow category is complete; all objects
have infinite arrows between them because there are infinite ways of adding two
numbers to two numbers and having them equate.


\todo{more examples plz}
\todo{what are the functors?}

\subsection{Slice Category}

The slice category $\cat{C}/X$, given $X \in \cat{C}$ is a special case of the
arrow category when $B = B' = X$.

\subsubsection{Principal Ideal}

Awodey gives the example of the slice $\cat{P}/P$ over a poset category for some
$P \in \cat{P}$, that this is isomorphic to the principal ideal
$\downarrow\!(P)$. What the heck does this mean? Well, let's draw the diagram:

$$\begin{tikzcd}
  A \cdrr{f} \cddr{!} & & A' \cddl{!} \\
  & P &
\end{tikzcd}$$

In a poset category, we can consider an arrow to be $\le$, therefore, in
$\cat{P}/P$ there is an arrow $\tfarr{f}{A}{A'}$ whenever $A \le A' \le P$.
Therefore, the principal ideal $\downarrow\!(P)$ is just the subset of the poset
that is $\le P$.

\subsubsection{Slice of a Monoid}

Out of curiosity, let's look at the slice $\;\cat{M}/M$ over some monoidal
category where $M \in \cat{M}$.

$$\begin{tikzcd}
  M \cdrr{f} \cddr{g_1} & & M \cddl{g_2} \\
  & M &
\end{tikzcd}$$

We can pull equations out of this diagram, namely that $g_1 = g_2 \circ f$. The
arrows in $\;\cat{M}/M$ are therefore the elements which can be "decomposed"
into the concatenation of two elements. However, because $\;\cat{M}$ is a
monoidal category, all elements can be decomposed (by factoring out a unit).
Therefore, $\;\cat{M}/M \cong\;\cat{M}$.

\subsubsection{Coslice Category}

We can define the coslice category $X/\cat{C}$, given $X \in \cat{C}$ by looking
at $(\cat{C}/X)^{Op}$. This is obviously a special case of the arrow category
when $A = A' = X$.

The coslice of a poset is its $\uparrow\!(X)$, and the coslice of a monoidal
category is still isomorphic to the category itself.

Awodey points out that $1/\cat{Set}$ is isomorphic to the category of pointed
sets, where each set has a distinguished member called the "point" (recall that
$\text{Maybe}$ is the free pointed set). Arrows in $1/\cat{Set}$ are
homomorphisms which preserve the points.

\subsection{Free Monoids}

Given a set $S = \{s_1, s_2, \dots, s_n \}$ of "letters", define $S^* =
\{\text{words over } S\}$, in other words, every finite sequence of letters in
$S$. The empty word can be denoted as -. $S^*$ forms a monoid under
concatenation with - as its unit.

We have an insertion function $i(s) = s : S \hookrightarrow S^*$. The elements
of $S$ generate $S^*$, because every $s \in S^*$ can be made up of some
concatenation of $s_{i1}s_{i2}\dots s_{in}$.

$S^*$ is said to be the free monoid over $S$ because it generates a monoid even
when $S$ itself is not one.

But we can state this definition more "categorically" by way of a
\textit{universal mapping property}.

The free monoid $M(A) \in \cat{Mon}$ over $A \in \cat{Set}$ is the object with
the property that for any $N \in \cat{Mon}$ the following diagram holds:

$$\begin{tikzcd}
  M(A) \cdrn{!} & N
\end{tikzcd}$$

\begin{theorem}
  $M(A) = A^*$
\begin{proof}
  We can prove the theorem by giving a program in Haskell.

  \begin{code}
foldMap :: Monoid n => (a -> n) -> [a] -> n
foldMap _ []       = mempty
foldMap f (a : as) = f a `mappend` foldMap f as
  \end{code}
\end{proof}
\end{theorem}

Awodey says something about how the monoid of natural numbers under addition is
isomorphic to the free monoid over a single element set. This is obviously true,
but I'm not entirely sure what his broader argument is.

\todo{page 28 -- determine what's going on here}

\subsubsection{Forgetful Functors}

For every structured set $Z$ (of category \cat{Z}) there is an underlying set
$\abs{Z}$, and likewise for every homomorphism $f$ over the structured set,
there is a corresponding homomorphism over sets $\abs{f}$. Therefore we have a
functor $\functor{U}{Z}{Set}$ called the forgetful functor.

\subsection{Free Categories}

Just like how $A^* \in \cat{Mon}$ is the free monoid over $A \in \cat{Set}$,
$Cat(G) \in \cat{Cat}$ is the free category over $G \in \cat{Graph}$ (the
category of directed graphs.)

A $Cat(G)$ can be generated from $G$ by taking every vertex $V \in G$ and making
it an object $V \in Cat(G)$. Every path $P$ made up of a finite sequence of
edges $E_1, E_2, \dots E_n \in G$ (where the target of $E_i$ is the source of
$E_{i+1}$) becomes an arrow in $Cat(G)$. Add the mandatory $\tfarr{1_V}{V}{V}$
identities, and you have yourself a category where composition of arrows is
subsequent traversals of paths in the underlying graph.

There is also \textit{universal mapping property} for free categories, namely
that for any category $\cat{C} \in \cat{Cat}$:

$$\begin{tikzcd}
  Cat(G) \cdrn{!} & \cat{C}
\end{tikzcd}$$

\todo{prove this}

\subsection{More Foundations}

A category is said to be "small" iff its objects and arrows both form sets.
Otherwise it is "large."

This implies that \cat{Set} and any other structured set categories are large,
and that \cat{Cat} is itself only the category of small categories -- and thus
does not contain itself.

However, a category \cat{C} can be said to be "locally small" iff for any
objects $X, Y \in \cat{C}$, the set of arrows $\{f \in \cat{C}\;|\;f : X \to Y
\}$ is a set. \cat{Set} is locally small because there is a set of all functions
from $X \to Y$, $Y^X$. Likewise, all other structured sets share this property,
since they are necessarily more restrictive than \cat{Set}.

Awodey asks whether \cat{Cat} is locally small. My intuition is yes, because all
of its objects are themselves small categories; therefore the collection of
functors from one small category to another must form a set.

\subsection{Exercises}

\begin{exercise}
  The objects of \cat{Rel} are sets, and an arrow \tfarr{f}{A}{B} is a relation
  from $A$ to $B$, that is, $f \subseteq A \times B$. The identity relation
  $\{\langle a, a\rangle \in A \times A \;|\; a \in A\}$ is the identity arrow
  on a set $A$.  Composition in \cat{Rel} is to be given by:

  $$
  g \circ f = \{\langle a, c\rangle \in A \times C \;|\; \exists b. \langle a,b
  \rangle \in f \;\&\; \langle b, c \rangle \in g\}
  $$

  Show that \cat{Rel} is a category.

  \begin{proof}
    We need to show that $\langle a, a\rangle$ is an identity, and that $f \circ
    (g \circ h) = (f \circ g) \circ h$.

    Given \tfarr{f}{A}{B}, we can show that $1_a = \langle a, a \rangle$ is an
    identity:

    \begin{align*}
      f \circ 1_a &=
  \{\langle a, b\rangle \in A \times B \;|\; \exists x. \langle x,x
  \rangle \in 1_a \;\&\; \langle x, b \rangle \in f\} \\
      &= \{\langle a, b\rangle \in A \times B \;|\; \langle a,a
  \rangle \in 1_a \;\&\; \langle a, b \rangle \in f\} \\
      &= \{\langle a, b\rangle \in A \times B \;|\;
  \langle a, b \rangle \in f\} \\
      &= f
    \end{align*}

    The proof that $1_b$ is a left identity proceeds in exactly the same way.
    Therefore, the identity relation is in fact an identity on the arrows.

    Finally, we need to show that composition is associative. This proceeds
    immediately from the existential in the definition of composition which
    asserts an equality between the (set) codomain of $f$ and the domain of $g$.
    Because equality is associative, composition in \cat{Rel} must too be.
  \end{proof}
\end{exercise}

\begin{exercise}
  Determine which of the following isomorphisms hold:

  \begin{enumerate}
    \item{$\cat{Rel} \cong \cat{Rel}^{op}$}
    \item{$\cat{Set} \cong \cat{Set}^{op}$}
    \item{For a fixed set $X$ with powerset $P(X)$, as poset categories $P(X)
      \cong P(x)^{op}$ (the arrows in $P(X)$ are subset inclusions $A \subseteq
      B$ for $A, B \subseteq X$)}
  \end{enumerate}

  \begin{proof}
    \quad
    \begin{enumerate}
    \item{$\cat{Rel} \cong \cat{Rel}^{op}$ because arrows compose due to an
      existential equality. Since equality is dual to itself, these two
      categories must be isomorphic.}
    \item{$\cat{Set} \not\cong \cat{Set}^{op}$ because in the dual, initial
      objects in \cat{Set} are mapped to terminal objects in $\cat{Set}^{op}$.
        Since isomorphisms need to preserve initial objects (and other
        interesting features of a category), these two categories are not
        isomorphic.}
    \item{\todo{Apparently this is in fact isomorphic to its dual, but I don't
      know why because it seems like the \cat{Set} argument should apply here as
        well.}}
    \end{enumerate}
  \end{proof}
\end{exercise}

\section{Abstract Structures}

\subsection{Epis and Monos}

A monomorphism (mono) is any morphism \tfarr[\mono]{f}{A}{B} such that for any
\tfarr{g, h}{C}{A} the following is true: $fg = fh \implies g = h$. In other
words, a mono is always left-cancelable.

$$\begin{tikzcd}
  C \arrow[shift right=2,swap]{r}{h} \arrow[shift left=1]{r}{g} & A \cdr{f} & B
\end{tikzcd}$$

Dually, an epimorphism (epi) is a morphism \tfarr[\epi]{f}{A}{B} such that for
any \tfarr{i, j}{B}{D} the following is true: $if = jf \implies i = j$. An epi
is always right-cancelable.

$$\begin{tikzcd}
  A \cdr{f} & B \arrow[shift right=2,swap]{r}{j} \arrow[shift left=1]{r}{i} & D
\end{tikzcd}$$

In \cat{Set}, monos correspond to injective functions, and epis are surjective
functions. Furthermore, in \cat{Set} a function which is monic and epic is an
iso, but this is \textbf{not true} in general.

\begin{theorem}
  In a poset \;\cat{P}, every arrow $p \leq q$ is both monic and epic.
  \begin{proof}
    In a poset, there is exactly one arrow between any two objects. Therefore if
    two arrows are equal after composing with a third, they must have been equal
    to begin with.
  \end{proof}
\end{theorem}

\begin{theorem}
  Every iso is a mono and an epi.
  \begin{proof}
    Given an iso \tfarr{m}{A}{B} with inverse \tfarr{e}{B}{A}:
    \begin{align*}
      mx &= my \\
      emx &= emy \\
      1_B x &= 1_B y \\
      x &= y
    \end{align*}

    And dually to show epicness.
  \end{proof}
\end{theorem}

\subsection{Initial and Terminal Objects}

An object $0 \in \cat{C}$ is said to be initial if for every other object $X \in
\cat{C}$ there is a unique arrow \tfarr{!}{0}{X}. Dually, a terminal object is
one which has a unique arrow coming into it from every object in the category.

\begin{theorem}
  Initial and terminal objects are unique up to isomorphism.
  \begin{proof}
    Assume we have two initial objects, $X$ and $Y$. In order to be initial,
    they must have arrows \tfarr{y}{X}{Y} and \tfarr{x}{Y}{X}. Furthermore, we
    know that there is exactly one arrow from an initial object to any other --
    including itself. By the category laws, this unique arrow must be 1, which
    makes the following diagram commute:

$$\begin{tikzcd}
  X \cdr{y} \cddr{1_X} & Y \cdd{x} \arrow{dr}{1_Y} \\
  & X \cdr{y} & Y
\end{tikzcd}$$

    This argument dualizes to terminal objects.

  \end{proof}
\end{theorem}

\subsection{Generalized Elements}

\subsubsection{Boolean Algebras}

A Boolean algebra is a poset $B$ with initial and terminal objects, products,
coproducts, and complements, with the laws:

\begin{enumerate}
  \item{$0 \leq a$ (initial object)}
  \item{$a \leq 1$ (terminal object)}
  \item{$a \leq c, b \leq c \iff a \vee b \leq c$ (coproducts)}
  \item{$c \leq a, c \leq b \iff c \leq a \wedge b$ (products)}
  \item{$a \leq \neg b \iff a \wedge b = 0 $ (complement)}
  \item{$\neg \neg a = a $ (complement)}
\end{enumerate}

Consider the case of a lattice:

$$\begin{tikzcd}
  & \setn{x, y, z} & \\
  \setn{x, y} \arrow{ur} & \setn{x, z} \arrow{u} & \setn{y, z} \arrow{ul} \\
  \setn{x} \arrow{u}\arrow{ur} & \setn{y}\arrow{ul}\arrow{ur} &
  \setn{z}\arrow{ul}\arrow{u} \\
  & \varnothing \arrow{ul}\arrow{u}\arrow{ur} &
\end{tikzcd}$$

We can define a complement $\neg x$ as $\setn{a\;|\;a \in B, a \not \in x}$.
This clearly satisfies the $\neg \neg x = x$ law, and some playing around with
the $a \leq \neg b \iff a \wedge b = 0$ law seems true, although I don't have a
proof for it.

\subsubsection{Ultrafilters}

A filter $F$ in a boolean algebra $B \in \cat{Bool}$ is a subset of $B$ which is
closed upwards and under meets (products). Which is to say:

\begin{enumerate}
  \item{$a \in F,\; a \leq b \implies b \in F$ (closed upwards)}
  \item{$a \in F,\; b \leq F \implies a \wedge b \in F$ (closed under meets)}
\end{enumerate}

An ultrafilter is a filter $F$ which is maximally big, ie. $F \subset F'
\implies F' = B$.

\begin{theorem}
  $F$ is an ultrafilter iff for all $x \in B$, either $x \in F$ or $\neg x \in
  F$, but not both.
  \begin{proof}
    Because a filter is closed upwards, any filter which contains $0$ must equal
    to $B$ itself, and thus not an ultrafilter.

    If $x \in F$ and $\neg x \in F$, then $x \wedge \neg x \in F$ by filters
    being closed under meets. However, $x \wedge \neg x = 0$, therefore, no
    ultrafilter can contain $x$ and $\neg x$.

    What if neither $x \in F$ nor $\neg x \in F$? This is true just if $F =
    \varnothing$, which is obviously not maximal.
  \end{proof}
\end{theorem}

Assume now we have (multiple) arrows \tfarr{p}{B}{2}, $p \in \cat{Bool}$. These
correspond exactly with the ultrafilters $U$ of $B$. We can get $p$ from $U$ by
defining $p_U(b) = 1 \text{ iff } b \in U$. Likewise, we can get $U$ from $p$ by
defining $U_p = p^{-1}(1)$ where $p^{-1}(1)$ is the domain of $p$ whose image is
$1$. These constructions obviously form an isomorphism.

The interesting part of all this jiggery-pokery is that $2 \in \cat{Bool}$ is an
initial object, and thus these functions $p$ are in fact arrows \textit{into} an
initial object. Awodey: "clearly only some of these will be of interest, but
those are often especially significant."

\subsubsection{Arrows out of Terminal Objects}

In \cat{Set}, $X \cong \text{Hom}_{\cat{Set}}(1, X)$. This is true because
$\abs{X^1} = \abs{X}$, and makes sense intuitively if you need to provide a
function from a singleton to $X$, you're going to have one for each element in
$X$. This is also true in \cat{Poset}. In any category with a terminal object,
these arrows $1 \to A$ are called the constants or points of $A$.

In \cat{Set} and \cat{Poset}, general arrows \tfarr{f}{A}{B} are uniquely
determined by their action on the points of $A$. Which is to say that if $fa =
f'a$ for all \tfarr{a}{1}{A}, then $f = f'$.

In \cat{Mon}, where 1 is the poset:

$$\begin{tikzcd}
  \bullet
\end{tikzcd}$$

there are obviously $\abs{P}$ points of the form \tfarr{p}{1}{P}. Therefore, if
two functions $f$ and $g$ agree on all the points, we must have $f = g$

\subsubsection{Generalized Elements}

However, this is not true of all categories in general. Take, for example,
\cat{Mon}. There is only one point \tfarr{p}{1}{P} which maps the unit in 1 to
the unit in $P$. Just because $f$ and $g$ agree on where they send the unit in
$P$ does not mean they agree on all the other elements of $P$.

We can generalize this idea of points being useful to the notion of
\textit{generalized elements}. A generalized element is one whose domain is not
1, but instead some arbitrary object $X \in \cat{C}$ .

For example, consider the posets:

$$\begin{tikzcd}
      & B & & C \\
  P = &   & A \arrow{ur} \arrow{ul} & \\
      & \quad & &
\end{tikzcd} \qquad \qquad \qquad
\begin{tikzcd}
     & Z \\
  Q = & Y \arrow{u} \\
     & X \arrow{u}
\end{tikzcd}$$

There is a bijective homomorphism \tfarr{f}{P}{Q} given by the map $f(A) \mapsto
X$, $f(B) \mapsto Y$, $f(C) \mapsto Z$. Despite this bijection, it is clear that
$P \not \cong Q$. We can show this via inspection of the general elements out of
2.

Generalized elements of the form \tfarr{g}{2}{P} are theorems of the shape $x
\leq y$ for $x \in P,\; y \in P$. If $P \cong Q$, they must agree on the number
of maps from $2$ to themselves. We can state this as $\abs{\hom{2}{P}} =
\abs{\hom{2}{Q}}$ is a necessary (though not \textit{sufficient}) condition of
$P \cong Q$.

However, $\abs{\hom{2}{P}} = 5$ (three identities, $A\leq B$ and $A\leq C$), but
$\abs{\hom{2}{Q}} = 6$ (three identities, $X \leq Y$, $Y \leq Z$ and $X\leq Z$.)
Because $P$ and $Q$ do not agree on the 2-elements, it must be the case that $P
\not \cong Q$.

Awodey mentions that $T$-elements (of form \tfarr{t}{T}{A}) are "figures of
shape $T$ in $A$." We have seen this when we noted that 2-elements were theorems
of the shape $x \leq y$ for $x \in A,\; y \in A$.

\subsection{Sections and Retractions}

\begin{theorem}
  An arrow \tfarr{f}{A}{B} has a left inverse \tfarr{g}{B}{A} (eg. $gf = 1_A$)
  if and only if $f$ be monic, and $g$ be epic.

  \begin{proof}
    \qquad
    \begin{align*}
      fx &= fy \\
      gfx &= gfy \\
      1_A x &= 1_A y \\
      x &= y
    \end{align*}
    This is the definition of a mono $f$, that for any $x,\; y$, $fx = fy
    \implies x = y$.

    Proving that $g$ is epic is equally trivial.
  \end{proof}
\end{theorem}

\subsubsection{Splits}

A split mono is an arrow with a left inverse. A split epi is one with a right
inverse. In the mono case:

\begin{align*}
  \tfarr{s}{A}{X} &\qquad \text{($s$ is a "splitting" of $e$)} \\
  \tfarr{e}{X}{A} &\qquad \text{($e$ is a "retraction" of $s$)} \\
  es = 1_A &
\end{align*}

$A$ is additionally called a "retract" of $X$. The following diagram might help:

$$\begin{tikzcd}
  A \arrow{r}{s} \arrow[swap]{dr}{1_A} & X \arrow{d}{e} \\
  & A
\end{tikzcd}$$

\todo{This terminology makes no sense to me. It seems like $X$ should be the
thing we call a retraction?}

While functors do not necessarily preserve monos and epis, they do in fact
preserve split monos and epis, since identity is an invariant under functors.

Awodey claims that "every epi splits in \cat{Set}" is equivalent to the axiom of
choice. I have no recourse but to believe him.

\subsection{Products}

The product diagram:

$$\begin{tikzcd}
  & X \arrow{dl} \arrow{dr} \arrow[dotted]{d}{!} & \\
  A & A\times B \arrow{l}{\pi_1} \arrow[swap]{r}{\pi_2} & B
\end{tikzcd}$$

\begin{theorem}
  Products are unique up to isomorphism.

  \begin{proof}
    If both $A \times B$ and $A \times' B$ are products, then the following
    diagram must commute:

$$\begin{tikzcd}
   & A\times B \arrow{dl} \arrow[swap]{dr} \arrow[dotted, swap, shift
   right=1]{dd}{!} \arrow[dotted, loop left]{}{!} & \\
  A & & B \\
   & A\times' B \arrow{ul} \arrow[swap]{ur} \arrow[dotted, swap, shift
   right=1]{uu}{!} \arrow[dotted, loop right]{}{!} &
\end{tikzcd}$$
  \end{proof}
\end{theorem}

\subsection{Hom-sets}

Recall that we can describe the set of all arrows in a category \cat{C} of the
form $A\to B$. This construction is called the hom-set, and is written as
\hom{A}{B}.

Now, given a function \tfarr{g}{B}{C}, we can produce the following arrow in
\cat{Set} called \tfarr{g_*}{\hom{A}{B}}{\hom{A}{C}}.

Such a thing defines a functor \functor{\hom{A}{-}}{C}{Set}, known as the.
covariant representable functor of $A$.

\begin{theorem}
  \hom{A}{-} is a functor $\cat{C}\to\cat{Set}$.

  \begin{proof}
    First, we will need to show that the functor preserves identities:
    $\hom{A}{1_X} = 1_{\hom{A}{X}}$

    Given \tfarr{f}{A}{X}, we have:

    \begin{align*}
      \hom{A}{1_X}(f) &= 1_X \circ f \\ &= f \\ &= 1_{\hom{A}{X}}(f)
    \end{align*}

    We also must show that it preserves composition, thus:

    \begin{align*}
      \hom{A}{h \circ g}(f) &= h \circ g \circ f \\ & =h \circ (g \circ f) \\ &=
      \hom{A}{h}(g \circ f) \\ &= \hom{A}{h}(\hom{A}{g}(f))
    \end{align*}
  \end{proof}
\end{theorem}

In \cat{Hask}, the covariant representable functor is the \texttt{Reader}
functor.

The construction is somewhat uninteresting, but hom-sets form products, in that
it is trivial to construct $\hom{X}{A} \times \hom{X}{B}$ with the obvious
product diagram. Awodey refers to this as $\upsilon_X$.

\begin{definition}
  A functor \functor{F}{C}{D} is said to \textbf{preserve binary products} iff
  the isomorphism $F(A\times B) \cong FA \times FB$ holds.
\end{definition}

\begin{theorem}
  If \cat{C} has products, \functor{\hom{X}{-}}{C}{Set} preserves products, for
  all $X \in \cat{C}$.

  \begin{proof}
    We want to show there is an isomorphism $\hom{X}{A\times B}\cong
    \hom{X}{A} \times \hom{X}{B}$.

    \todo{we can just show a bijection via cardinality since these things are
    all in set}
  \end{proof}
\end{theorem}


\subsection{Exercises}

\begin{exercise}
  A function between sets is surjective if it is an epimorphism in \cat{Set}.

  \begin{proof}
    We want to show that an epimorphism in \cat{Set} cashes out as a surjective
    function between two sets.

    An epimorphism is right-cancelable, which is to say that for an epi
    \tfarr{e}{A}{B},
    $f\circ e = g\circ e$ implies $f = g$.

    By way of contradiction, $f$ and $g$ might agree on all inputs except $x$.
    If $x$ is not in the range of $e$, even though $f\circ e = g\circ e$, by
    construction $f \not = g$. Therefore, the range of $e$ must be the entire
    set $B$, which is the definition of a surjective function.
  \end{proof}
\end{exercise}

\begin{exercise}
  Given the commutative triangle:

$$\begin{tikzcd}
  A \arrow{r}{f} \arrow[swap]{dr}{h} & B \arrow{d}{g} \\
  & C
\end{tikzcd}$$

\begin{enumerate}
\item{
    Show that $h$ is an iso if $f$ and $g$ are too.

    \begin{proof}
      If $f$ and $g$ are isos, we must have $f^{-1}$ and $g^{-1}$ with the
      obvious identity equalities. Because the triangle commutes, $h = g\circ
      f$, therefore we can form $h^{-1} = f^{-1}\circ g^{-1}$.
    \end{proof}}

\item{
    Show that $f$ is monic if $h$ is monic.

    \begin{proof}
      If $h$ is monic, it means it is left-cancelable. Which means if we had
      some other functions \tfarr{x, y}{X}{A}:

$$\begin{tikzcd}
  X \arrow[shift left=2]{d}{y} \arrow[swap, shift right=2]{d}{x} & \\
  A \arrow{r}{f} \arrow[swap]{dr}{h} & B \arrow{d}{g} \\
  & C
\end{tikzcd}$$

      Now, we want to know whether $fx = fy$ implies $x = y$. Well, if we have
      $fx = fy$, we can compose with $g$ to get $gfx = gfy$. But $gf = h$, so we
      have $hx = hy$, and because $h$ is monic, $x = y$.
    \end{proof}}

\item{
    Show that $g$ is epic if $h$ is epic.
    \begin{proof}
      Trivially dual to the previous exercise.
    \end{proof}}

\item{
    Show that just because $h$ be monic, $g$ need not be.
    \begin{proof}
      Let $g(x) = x\quad(\text{mod } 2)$, and $f(x) = 2x$. Therefore, $h(x) =
      0$, which is obviously monic, but $g$ is obviously not.
    \end{proof}}
\end{enumerate}
\end{exercise}


\begin{exercise}
  Show that for any Boolean algebra $B$, Boolean homomorphisms \tfarr{h}{B}{2}
  correspond exactly to ultrafilters in $B$.
  \begin{proof}
    \todo{}
  \end{proof}
\end{exercise}

\begin{exercise}
  In any category with binary products, show that $A\times (B\times C) \cong
  (A\times B)\times C$

  \begin{proof}
    By diagram chasing:

$$\begin{tikzcd}
  & A\times (B\times C) \arrow{dl} \arrow[dashed]{dd} \arrow[dashed, shift right=2]{ddr} \arrow{drr} & & \\
  A & & B & B \times C \arrow{l} \arrow{d} \\
  & A\times B \arrow{ul} \arrow{ur} & (A \times B) \times C \arrow[dashed,
  shift right=2]{uul} \arrow{l} \arrow{r} \arrow[dashed]{ur} & C
\end{tikzcd}$$
  \end{proof}
\end{exercise}


\section{Duality}

\subsection{Poset Coproducts}

\todo{how does this work?}

\subsubsection{Rooted Poset Coproducts}

Consider the category \cat{Poset_0} of posets that have a initial object.  We
can form a more interesting coproduct here by equating their bottom elements,
and then overlaying the posets on top of one another.

$$
A +_{\cat{Poset_0}} \!B = (A +_{\cat{Poset}} \!B)/(0_A = 0_B)
$$

$$\begin{tikzcd}
  A &   & X\arrow{rr} &   & Y &   & A & X\arrow{r} & Y \\
    & +_{\cat{Poset_0}} &   &   &   & = &   &   &   \\
  0\arrow{uu} &   &   & 0\arrow{uul}\arrow{uur} &   &   &   & 0
  \arrow{uul}\arrow{uu}\arrow{uur} &   \\
\end{tikzcd}
$$

\subsection{Monoid Coproducts}

\begin{code}
monoidCoproduct :: (Monoid a, Monoid b) => [Either a b] -> [Either a b]
monoidCoproduct = filter (/= mempty)
                . join
                . fmap mconcat
                . groupBy (compare `on` isLeft)
\end{code}

This doesn't exactly capture the notion in \cat{Hask} due to weird functor
instances, but essentially, you want to \texttt{mappend} subsequent
\texttt{Left} and \texttt{Right} values, and then remove any groups that are
just \texttt{mempty}.

\subsection{Abelian Groups}

\todo{what? awodey talks about free products, and how they need not be
abelian. but then he says they need to be abelian so wtf}

The construction here is pretty neat; refer to page 63.

\subsection{Equalizers}

In any category \cat{C}, given arrows \tfarr{f,g}{A}{B}

$$\begin{tikzcd}
  A \arrow[shift right=2,swap]{r}{g} \arrow[shift left=1]{r}{f} & B
\end{tikzcd}$$

an \textit{equalizer} of $f$ and $g$ consists of an object $E$ and an arrow
\tfarr{e}{E}{A} such that $f\circ e = g\circ e$.

Which is to say, the following diagram commutes:

$$\begin{tikzcd}
  E\arrow{r}{e} & A \arrow[shift right=1,swap]{r}{g} \arrow[shift left=1]{r}{f} & B \\
  & & \\
  Z \arrow[dashed]{uu}{u} \arrow[swap]{uur}{z} & &
\end{tikzcd}$$

For any arbitrary \tfarr{z}{Z}{A}, there is a unique \tfarr{u}{Z}{E} such that
$z = e\circ u$. But what does this mean??

\subsubsection{Equalizers in \cat{Set}}

An equalizer in \cat{Set} between $f$ and $g$ is the contrived set $E =
\setn{x\;|\; x \in A,\;f(x) = g(x)}$, and \tfarr[\hookrightarrow]{e}{E}{A}.

\todo{what the shit does any of this mean?} "Since if $fh(z) = gh(z)$ for some
\tfarr{h}{Z}{A}, then $h(z) \in E$ for all $z \in Z$, whence $h$ "factors
through" the inclusion function $i$, in the sense that there is a function
\tfarr{\bar{h}}{Z}{E} such that $i\circ\bar{h} = h$."

\end{document}

