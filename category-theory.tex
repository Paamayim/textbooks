\documentclass[]{article}
\usepackage{amsmath,amscd}
\usepackage{amssymb}
\usepackage{amsthm}
\usepackage{todonotes}
\usepackage{tikz-cd}
\usepackage{calligra}
\DeclareMathAlphabet{\catty}{T1}{calligra}{m}{n}

\usepackage [english]{babel}
\usepackage [autostyle, english = american]{csquotes}
\usepackage{fancyvrb}
\DefineVerbatimEnvironment{code}{Verbatim}{fontsize=\small}
\MakeOuterQuote{"}

\newtheorem{theorem}{Theorem}
\newtheorem{definition}{Definition}

\newcommand{\what}{}
% \newcommand{\cat}[1]{\mathbf{#1}}
\newcommand{\defn}[2]{
\renewcommand{\what}{\textit{#1} }
\textbf{Definition:} #2\\
}
\newcommand{\diag}[1]{$$\begin{CD}#1\end{CD}$$}
\newcommand{\cdr}[1]{\arrow[swap]{r}{#1}}
\newcommand{\cdrn}[1]{\arrow[swap,dotted]{r}{#1}}
\newcommand{\cdrr}[1]{\arrow[swap]{rr}{#1}}
\newcommand{\cddl}[1]{\arrow{dl}{#1}}
\newcommand{\cddr}[1]{\arrow[swap]{dr}{#1}}
\newcommand{\cdl}[1]{\arrow{l}{#1}}
\newcommand{\cdd}[1]{\arrow{d}{#1}}
\newcommand{\cdu}[1]{\arrow{u}{#1}}

\newcommand{\abs}[1]{\left|#1\right|}

\newcommand{\tfarr}[3]{\ensuremath{#1 : #2 \to #3}}
\newcommand{\functor}[3]{\ensuremath{#1 : \cat{#2} \to \cat{#3}}}
\newcommand{\cat}[1]{\ensuremath{\!\! \catty{#1} \,\,}}

%opening
\title{Category Theory}
\author{Sandy Maguire}

\begin{document}

\maketitle

\begin{abstract}
My notes summarizing Awodey for the purposes of learning Category Theory. It's
going to be a great project.
\end{abstract}

\newpage

\section{Foundations}

\subsection{Definition of a Category}

A category \cat{C} consists of \textbf{objects} and \textbf{arrows} between
them. To be precise, every arrow has a domain and a codomain, both of which are
objects in the category. In addition, every object $X \in \cat{C}$ has an
\textbf{identity arrow} \tfarr{1_X}{X}{X}.

If \tfarr{f}{A}{B}, we say $dom(f) = A$ and $cod(f) = B$.

In addition, to be a category, \cat{C} must respect the following laws:

\begin{enumerate}
  \item{Composition: If \tfarr{f}{A}{B} and \tfarr{g}{B}{C}, there exists an
    arrow \tfarr{g \circ f}{A}{C}.}
  \item{Associative: $f \circ (g \circ h) = (f \circ g) \circ h$}.
  \item{Identity: $1_B \circ f = f = f \circ 1_A$, given \tfarr{f}{A}{B}}
\end{enumerate}

Anything that satisfies these laws is a category. It need not correspond to our
intuitions that "arrows are functions" or any such silliness.

\subsection{Definition of a Functor}

A functor \functor{F}{C}{D} is a mapping of objects in \cat{C} to objects in
\cat{D}, and likewise for arrows. A functor is a homomorphism across domains,
codomains and compositions.

That is to say:

\begin{enumerate}
  \item{$F(\tfarr{f}{A}{B}) = \tfarr{F(f)}{F(A)}{F(B)}$}
  \item{$F(g\circ f) = F(g) \circ F(f)$}
  \item{$F(1_A) = 1_{F(A)}$}
\end{enumerate}

There is an identity functor \functor{1_{\cat{C}}}{C}{C}, and because functors
compose, we have a category of categories: \cat{Cat}.

\subsection{Definition of an Isomorphism}

An arrow \tfarr{f}{A}{B} is called an isomorphism if there exists an arrow
\tfarr{g}{B}{A} such that $g \circ f = 1_A$ and $f \circ g = 1_B$.

\begin{theorem}
Isomorphisms are unique.
\begin{proof}
  For \tfarr{f}{A}{B} to be an isomorphism, we must have \tfarr{g}{B}{A}. Assume
  that the isomorphism is not unique, and thus that we also have
  \tfarr{g'}{B}{A}.

  By definition, we have $g \circ f = 1_A$. We can compose on both sides to get
  $g \circ f \circ g' = 1_A \circ g'$, but recall that $g'$ is an isomorphism,
  therefore $g \circ 1_B = 1_A \circ g'$. We can emit the identities, and thus
  $g = g'$.
\end{proof}
\end{theorem}

\begin{definition}
  A \textbf{group} is a single object category where every arrow is an
  isomorphism.
\end{definition}

\begin{theorem}
Every group $G$ is isomorphic to a group of permutations.
\begin{proof}
  Define $f_g(x) = g \times x$ for $g \in G$. Since $G$ is a group, we also have
  $f_{g^{-1}}(x) = g^{-1} \times x = f_g^{-1}(x)$ which means $(f_g \circ
  f_g^{-1})(x) = (f_g^{-1} \circ f_g)(x) = x$. Therefore $f_g$ forms a group.

  Consider now a function $\tfarr{T}{G}{\bar{G}}$ where $T(g) = f_g$.  $T$ is a
  group homomorphism because $(f_g \circ f_h)(x) = f_g(f_h(x)) = f_g(h * x) = g
  * (h * x) = f_{g*h}(x)$.

  This is true for any $x$, therefore $T(g) \circ T(h) = f_g \circ f_h = f_{g*h}
  = T(g * h)$
\end{proof}
\end{theorem}

\section{Constructions on Categories}

\subsection{Product Category}

The product of categories \cat{C} and \cat{D} is $\cat{C}\times\cat{D}$. It's
objects are the cartesian product of objects in \cat{C} and \cat{D}. Arrows are
likewise defined in this matter, with composition and units being defined
component-wise:

\begin{align*}
  1_{(C, D)} &= (1_C, \; 1_D) \\
  (f', g') \circ (f, g) &= (f' \circ f, \; g' \circ g)
\end{align*}

There are also projection functors
$\functor{\pi_1}{\cat{C}\times\cat{D}}{\cat{C}}$ and
$\functor{\pi_2}{\cat{C}\times\cat{D}}{\cat{C}}$ defined in the obvious way.

\subsection{Arrow Category}

The arrow category $\cat{C}^\rightarrow$ has objects which are arrows in \cat{C}
and its arrows are commutative squares in \cat{C}.

For example, given $f,\;f',\;g_1,\;g_2 \in \cat{C}$, there is an arrow $\tfarr{g
}{f}{f'} \in \cat{C}^\rightarrow$ such that $g = (g_1,\;g_2)$ if the follow
diagram exists in \cat{C}:

$$\begin{tikzcd}
  A \cdr{g_1} \cdd{f} & A' \cdd{f'} \\
  B \cdr{g_2} & B'
\end{tikzcd}$$

Composition acts as you'd expect. Given $h \circ g$, we get the diagram:

$$\begin{tikzcd}
  A \cdr{g_1} \cdd{f} & A' \cdr{h_1} \cdd{f'} & A'' \cdd{f''}  \\
  B \cdr{g_2} & B' \cdr{h_2} & B''
\end{tikzcd}$$

In other words, there is an arrow $f\to f'$ iff $g_2\circ f = f' \circ g_1$. But
what does this mean?

In a monoid, composition means concatenation, and thus over a free monoid of the
alphabet, $f = \text{art}$, $f' = \text{far}$, $g = (\text{t}, \text{f})$
because $\text{f}\circ\text{art} = \text{far}\circ\text{t}$. With the monoid of
addition over the natural numbers, the arrow category is complete; all objects
have infinite arrows between them because there are infinite ways of adding two
numbers to two numbers and having them equate.


\todo{more examples plz}
\todo{what are the functors?}

\subsection{Slice Category}

The slice category $\cat{C}/X$, given $X \in \cat{C}$ is a special case of the
arrow category when $B = B' = X$.

\subsubsection{Principal Ideal}

Awodey gives the example of the slice $\cat{P}/P$ over a poset category for some
$P \in \cat{P}$, that this is isomorphic to the principal ideal
$\downarrow\!(P)$. What the heck does this mean? Well, let's draw the diagram:
cdd
$$\begin{tikzcd}
  A \cdrr{f} \cddr{!} & & A' \cddl{!} \\
  & P &
\end{tikzcd}$$

In a poset category, we can consider an arrow to be $\le$, therefore, in
$\cat{P}/P$ there is an arrow $\tfarr{f}{A}{A'}$ whenever $A \le A' \le P$.
Therefore, the principal ideal $\downarrow\!(P)$ is just the subset of the poset
that is $\le P$.

\subsubsection{Slice of a Monoid}

Out of curiosity, let's look at the slice $\;\cat{M}/M$ over some monoidal
category where $M \in \cat{M}$.

$$\begin{tikzcd}
  M \cdrr{f} \cddr{g_1} & & M \cddl{g_2} \\
  & M &
\end{tikzcd}$$

We can pull equations out of this diagram, namely that $g_1 = g_2 \circ f$. The
arrows in $\;\cat{M}/M$ are therefore the elements which can be "decomposed"
into the concatenation of two elements. However, because $\;\cat{M}$ is a
monoidal category, all elements can be decomposed (by factoring out a unit).
Therefore, $\;\cat{M}/M \cong \cat{M}$.

\subsubsection{Coslice Category}

We can define the coslice category $X/\cat{C}$, given $X \in \cat{C}$ by looking
at $(\cat{C}/X)^{Op}$. This is obviously a special case of the arrow category
when $A = A' = X$.

The coslice of a poset is its $\uparrow\!(X)$, and the coslice of a monoidal
category is still isomorphic to the category itself.

Awodey points out that $1/\cat{Set}$ is isomorphic to the category of pointed
sets, where each set has a distinguished member called the "point" (recall that
$\text{Maybe}$ is the free pointed set). Arrows in $1/\cat{Set}$ are
homomorphisms which preserve the points.

\subsection{Free Monoids}

Given a set $S = \{s_1, s_2, \dots, s_n \}$ of "letters", define $S^* =
\{\text{words over } S\}$, in other words, every finite sequence of letters in
$S$. The empty word can be denoted as -. $S^*$ forms a monoid under
concatenation with - as its unit.

We have an insertion function $i(s) = s : S \hookrightarrow S^*$. The elements
of $S$ generate $S^*$, because every $s \in S^*$ can be made up of some
concatenation of $s_{i1}s_{i2}\dots s_{in}$.

$S^*$ is said to be the free monoid over $S$ because it generates a monoid even
when $S$ itself is not one.

But we can state this definition more "categorically" by way of a
\textit{universal mapping property}.

The free monoid $M(A) \in \cat{Mon}$ over $A \in \cat{Set}$ is the object with
the property that for any $N \in \cat{Mon}$ the following diagram holds:

$$\begin{tikzcd}
  M(A) \cdrn{!} & N
\end{tikzcd}$$

\begin{theorem}
  $M(A) = A^*$
\begin{proof}
  We can prove the theorem by giving a program in Haskell.

  \begin{code}
foldMap :: Monoid n => (a -> n) -> [a] -> n
foldMap _ []       = mempty
foldMap f (a : as) = f a `mappend` foldMap f as
  \end{code}
\end{proof}
\end{theorem}

Awodey says something about how the monoid of natural numbers under addition is
isomorphic to the free monoid over a single element set. This is obviously true,
but I'm not entirely sure what his broader argument is.

\todo{page 28 -- determine what's going on here}

\subsubsection{Forgetful Functors}

For every structured set $Z$ (of category \cat{Z}) there is an underlying set
$\abs{Z}$, and likewise for every homomorphism $f$ over the structured set,
there is a corresponding homomorphism over sets $\abs{f}$. Therefore we have a
functor $\functor{U}{Z}{Set}$ called the forgetful functor.

\subsection{Free Categories}

Just like how $A^* \in \cat{Mon}$ is the free monoid over $A \in \cat{Set}$,
$Cat(G) \in \cat{Cat}$ is the free category over $G \in \cat{Graph}$ (the
category of directed graphs.)

A $Cat(G)$ can be generated from $G$ by taking every vertex $V \in G$ and making
it an object $V \in Cat(G)$. Every path $P$ made up of a finite sequence of
edges $E_1, E_2, \dots E_n \in G$ (where the target of $E_i$ is the source of
$E_{i+1}$) becomes an arrow in $Cat(G)$. Add the mandatory $\tfarr{1_V}{V}{V}$
identities, and you have yourself a category where composition of arrows is
subsequent traversals of paths in the underlying graph.

There is also \textit{universal mapping property} for free categories, namely
that for any category $\cat{C} \in \cat{Cat}$:

$$\begin{tikzcd}
  Cat(G) \cdrn{!} & \cat{C}
\end{tikzcd}$$

\todo{prove this}

\end{document}
